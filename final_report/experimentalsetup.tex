%%%%%%%%%%%%%%%%%%%%%%%%%%%%%%%%%%%%%%%%%%%%%%%%%%%%%%%%%%%%%%%%%%%%%%%%%%%%%%%%
%% Experimental Setup:
%%  - What our methodology was
%%  - Details about the Phones
%%  - Some information about NIST
%%
%%%%% Replacing these as well:
%%% \section{Plan for Development}
%%% %% Tasks


%%% 
%%% \subsection{Risks and Unknowns}
%%% %% Risks and Unknowns

% Talk about:
%  - Is the accelerometer really entropic?
%  - Will this be worthwhile vs. the hw-rng?
%  - Is this data being incorporated somehow already?
%  - Unknown whether post processing is ``good''.

For any project, it is important to understand that there are risks and unknowns
that will need to be understood and mitigated during the course of the project.
By assessing what may become a problem during the process of the project, those
problems can be consciously avoided or dealt with early on.

One unknown that is central to this project is the entropy of the accelerometer.
Based on the findings in \cite{voris}, it appears that accelerometers are good
sources of entropy, it is possible that the accelerometer on the Dragonboard
violates some assumption or that technology has changed such that modern
accelerometers are no longer good sources of entropy. This can be tested using
several randomness tests, such as the DIEHARD and NIST entropy tests, which will
be essential to do before moving on with any sort of analysis.  These tests will
show if the sensor data appears to have good entropy or not.  In the case that
it does not, it would invalidate the remaining work for the project. 

A potential problem is the existence of the hardware random number generator
that is already on the board. Depending on how good that hardware is, it could
potentially dwarf any results that this project generates. Even so, the results
should be valuable, both as a look at the prospect of using accelerometers in
future designs for hardware random number generators as well as for providing
better entropy to existing platforms which do not have hardware random number
generators. 

During the testing phase of the project, members of the project team will
perform an experiment to show what sort of data is experienced over the course
of a couple of days. As it is not reasonable to carry the Dragonboard over the
course of a daily routine, some of the data will need to be collected on
personal cell phones. It could potentially be a problem if no available cell
phone has a similar accelerometer or gyroscope as this would introduce an
additional variable into the analysis. Research and experimentation will have to
be done to discover if, and show, the sensors in the selected cell phones will
be a reasonable stand in for the Dragonboard. If nothing matches closely enough,
a different method of collecting longer term real world data will have to be
devised.

If the data from these sensors is already being incorporated into the entropy
pool, it would make this work largely a repetition of that previous work. Upon
first inspection this does not appear to be the case, though further research is
needed to ensure that this is indeed novel work.

%%% 
%%% \section{Deliverables}
%%% % Deliverables

The deliverables of this project will include a detailed report, the 
daemon source, post processing source, and any scripts used in 
development. Additionally, the results will be presented, summarizing 
the findings in the report.  

The results of this work should show that the potential exists to use 
common peripherals on mobile devices to help fill the entropy pool of 
the RNG. This will be beneficial to developers who want to ensure a 
consistent stream of high-quality random numbers. Even though the 
end-user of these devices may not be aware of the RNG, they will notice 
slow downs in encrypted application sessions. The proposed solution 
should reduce the chances of end-users experiencing slow downs when 
concurrent encrypted sessions require random numbers quickly.

%%%%%%%%%%%%%%%%%%%%%%%%%%%%%%%%%%%%%%%%%%%%%%%%%%%%%%%%%%%%%%%%%%%%%%%%%%%%%%%%

In order to show that sampling accelerometer and gyroscope data is an effective
mechanism for randomness generation on mobile phones, a testing methodology was
developed. This testing methodology involved gathering data from mobile phones
and utilizing well regarded statistical analysis suites to determine the quality
of the results. To maintain data equivalence across testing, a single platform
was chosen for all of the tests. 

%%% Information about the Phone
%%%   - Specs
%%%   - Why it was chosen over other phones... lol
\subsection{Samsung Nexus S}

The platform used for testing was the Samsung Nexus S. The Nexus S was first
released in December of 2010. It uses a Samsung Exynos 3110 System on Chip
(SoC). The Exynos 3110 is a single core ARM Cortex-A8 using the ARMv7
instruction set. In the Nexus S, the processor operates at a 1GHz clock
frequency and is paired with 512MB of main memory. The phone originally shipped
with Android 2.3 Gingerbread installed but was updated to the latest supported
build, 4.1.2 Jelly Bean.  The Nexus S has both a three-axis gyroscope and
accelerometer provided by the InvenSense MPU-6050. The MPU-6050 communicates to
the CPU via an I\textsuperscript{2}C bus operating at 400KHz. 

The Nexus S was selected because it represented both a vanilla Android mobile
phone, but also because it is an example of a widely popular mobile phone that
does not have a hardware random number generator. As such it was a good first
target for implementing an additional source of entropy for the kernel's entropy
pool. Ideally, by providing fast random number generation on a platform that
suffers from a reduced number of sources of entropy, improved cryptographic
speed as well as security could be brought to users of such devices.


%%% Background about NIST
\subsection{NIST Statistical Testing Suite}

The National Institute of Standards and Technology (NIST) publishes a
statistical testing suite for the verification of random number generators.
Described in Special Publication 800-22 \cite{nist}, the \textit{Statistical
Test Suite for Random and Pseudorandom Number Generators for Cryptographic
Applications} comprises 15 statistical tests that analyze a source of bitstreams
for the apparent lack of entropy. The documentation does note both that some
tests are expected to fail even for good pseudorandom number generators and it
should further be noted that even a random number generator that appears to pass
these tests is not necessarily random, it just has the appearance thereof.

The NIST statistical test suite was selected for use because it provides a
rigorous testing suite that is recognized as providing quality statistical
analysis of apparently random bitstreams. This statistical test suite was used
to test the quality of potentially random data from several sources, including
unprocessed data from the sensors, random devices from the stock system, as well
as several modifications on adding data from the sensors to the kernel's entropy
pool and drawing from the randomness devices.

% NIST-> How does it work and how do we get it.
NIST takes a binary input file and allows the user to define the length of the
bitstream as well the number of bitstreams pulled from the file. After that is
defined, the software will drive several statistical tests against the data.
Each test defined what it means to ``pass'' the test for a given bitstream. The
NIST statistical test suite then returns the proportion of tests that passed.
For the testing done in this work, 10 bitstreams were used. For a 10 bitstream
test, a passing random number generator is expected to have at least a 9 of the
10 tests passing.

%%% What was developed in order to perform the testing?
\subsection{Data Collection Software}

In order to test the efficacy of using accelerometer and gyroscope sampling to
feed the entropy pool, several pieces of software had to be developed to aid in
the research. The software developed is split into two distinct applications
that work together: the java application, \textit{PollSensors} and the entropy
insertion daemon, \textit{entd}. This model was chosen because of limitations in
the security model of the Android Operating System. Being able to add entropy to
the kernel's pool is a privileged operation to prevent attackers from poisoning
the host's randomness, but Android applications can only launch as a user
process and then execute a privileged native application, such as \textit{su}.
As a result of these requirements, the entropy insertion daemon is written as a
native application that runs with privileged authority and the entropy
collection is written as an Android application that communicates with the
entropy insertion daemon using a named pipe. This model allows for working
within the confines of the security model of the Dalvik VM while performing
privileged actions securely from a privileged native application.

%%% The Testing Methodology:
%%%  - What was done
%%%  - What data was gathered
%%%  - How was it analyzed 
\subsection{Methodology}

Data collection was done using the \textit{PollSensors} and \textit{entd}
software on two Samsung Nexus S phones. Data collection was timed, and the goal
was to collect at least 10 MB worth of binary data for analysis, though due to
the speed at which certain devices operated this was not always possible. The
purpose of timing the tests was to be able to calculate the average bitrate of
the various randomness sources.

Data was collected from the kernel's stock \textit{random} and \textit{urandom}
devices, raw data from the gyroscope and accelerometer, bitfiltered versions of
that data, von Neumann whitened versions of that data, as well as from the
\textit{random} device after raw and von Neumann whitened data had been passed
into the kernel's entropy pool. Once the data had been collected, the NIST
statistical analysis software was used to determine if it had the statistical
qualities of random data or not.

