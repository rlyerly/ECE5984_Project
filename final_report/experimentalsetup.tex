%%%%%%%%%%%%%%%%%%%%%%%%%%%%%%%%%%%%%%%%%%%%%%%%%%%%%%%%%%%%%%%%%%%%%%%%%%%%%%%%
%% Experimental Setup:
%%  - What our methodology was
%%  - Details about the Phones
%%  - Some information about NIST
%%%%%%%%%%%%%%%%%%%%%%%%%%%%%%%%%%%%%%%%%%%%%%%%%%%%%%%%%%%%%%%%%%%%%%%%%%%%%%%%

In order to show that sampling accelerometer and gyroscope data is an effective
mechanism for randomness generation on mobile phones, a testing methodology was
developed. This testing methodology involved gathering data from mobile phones
and utilizing well regarded statistical analysis suites to determine the quality
of the results. To maintain data equivalence across testing, a single platform
was chosen for all of the tests. 

%%% Information about the Phone
%%%   - Specs
%%%   - Why it was chosen over other phones... lol
\subsection{Samsung Nexus S}

The platform used for testing was the Samsung Nexus S. The Nexus S was first
released in December of 2010. It uses a Samsung Exynos 3110 System on Chip
(SoC). The Exynos 3110 is a single core ARM Cortex-A8 using the ARMv7
instruction set. In the Nexus S, the processor operates at a 1GHz clock
frequency and is paired with 512MB of main memory. The phone originally shipped
with Android 2.3 installed but was updated to the latest supported build, 4.1.2.
The Nexus S has both a three-axis gyroscope and accelerometer provided by the
InvenSense MPU-6050. The MPU-6050 communicates to the CPU via an
I\textsuperscript{2}C bus operating at 400KHz. 

The Nexus S was selected because it represented both a vanilla Android mobile
phone, but also because it is an example of a widely popular mobile phone that
does not have a hardware random number generator. As such it was a good first
target for implementing an additional source of entropy for the kernel's entropy
pool.


%%% The Testing Methodology:
%%%  - What was done
%%%  - What data was gathered
%%%  - How was it analyzed 
\subsection{Methodology}

