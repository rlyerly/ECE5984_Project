%%%%%%%%%%%%%%%%%%%%%%%%%%%%%%%%%%%%%%%%%%%%%%%%%%%%%%%%%%%%%%%%%%%%%%%%%%%%%%%%
%% Conclusion:
%%      - Gryoscope and Accel. Data alone is not good enough
%%      - Significant post processing or dedicated hardware
%%      - Call for more dedicated TRNG's   
%%    
%%%%%%%%%%%%%%%%%%%%%%%%%%%%%%%%%%%%%%%%%%%%%%%%%%%%%%%%%%%%%%%%%%%%%%%%%%%%%%%%

    The statistics produced from the NIST testsuite lead us to conclude that the accelerometer 
and gyroscope alone are not good sources of entropy for our target platform. However, pure random sources are nearly
impossible to find without the use of sufficient hashing. In our test case we found that these two 
 sources would need a substantial amount more post processing than the Von Neumann 
whitening technique provides. We also found that when blending any of our bit streams into 
the random device's entropy pool, the internal hashing provided enough distribution to pass NIST's statistical
tests. The blending into the pool also provided a significant boost to the bitrate of the random device.

    The current trend in the mobile phone hardware community is to package a hardware random
number generators into the phone. These devices are very well tested for producing randomness in low power
conditions. As for randomness on legacy devices, we found that the
\textit{urandom} device will produce large 
amounts of data rapidly, the data produced will pass a suite of statistical benchmarks. Therefore, this may be sufficient for
many applications running on Android devices. In conclusion, for access to true random numbers on mobile phones, there
needs to be either a substantial amount of post processing or a dedicated piece of hardware for random number generation.

