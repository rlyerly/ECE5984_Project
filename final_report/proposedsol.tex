Although mobile phones have many different devices on-chip that could be
potentially used as sources of randomness for a random number generator, most of
these devices have the potential for strong bias.  Additionally, because of the
operating environment, polling these devices with high frequency can cause
serious system performance and energy issues.  Therefore, it is necessary to not
only use devices that have high quality randomness, but to use those devices in
an intelligent manner so as to not overwhelm the computational and energy
resources of the device.

We propose the implementation of a Android process that polls data from the
accelerometer and gyroscope (when the entropy pool is running low), processes
input data to eliminate sources of bias, and feeds the data into the entropy
pool.  Fig.~\ref{proposed_solution_block} shows the architecture for our
application. Android built on Linux exposes standard interfaces to the hardware
sensors for data retrieval; additionally, Linux provides system calls to feed
random data into the entropy pool.  Our application will interact with both of
these interfaces.

%There are many types of post processing that can be implemented to increase the
%uniformity of the random numbers generated.  Hash algorithms like SHA-3
%\cite{keccak} can be used to eliminate bias.  Similarly, Barak et al. propose a
%technique whereby random numbers are pulled from several independent devices
%\cite{independent_devices}.  Two of these numbers are multiplied together and
%added to a third number to strengthen the randomness.  Barak et al. propose
%another technique where random numbers are multiplied by a Toeplitz matrix to
%approach a uniform distribution \cite{true_rng}.  These techniques will be
%implemented and compared for their effectiveness and computational complexity.
%
%Random numbers generated by our daemon will be compared to the stock
%Android/Linux random number generator (which includes an on-board hardware RNG
%that may not be present on all mobile devices).  These two sources will be
%compared on three dimensions: quality of random numbers generated, energy draw
%from using the random number generator, and computational resources used by
%random number generation.  The DIEHARD \cite{diehard} / NIST \cite{nist}
%benchmarks will be used to evaluate the quality of the random numbers
%generated; synthetic benchmarks will be written to evaluate the energy draw and
%computational complexity of the random number generators.

%just mention V. N. Whitening here?

%There are many types of post processing that can be implemented to increase the
%uniformity of the random numbers generated.  Hash algorithms like SHA-3
%\cite{keccak} can be used to eliminate bias.  Similarly, Barak et al. propose a
%technique whereby random numbers are pulled from several independent devices
%\cite{independent_devices}.  Two of these numbers are multiplied together and
%added to a third number to strengthen the randomness.  Barak et al. propose
%another technique where random numbers are multiplied by a Toeplitz matrix to
%approach a uniform distribution \cite{true_rng}.  These techniques will be
%implemented and compared for their effectiveness and computational complexity.

There are many types of post processing that can be implemented to increase the
uniformity of the random numbers generated. The post-processing proposed for the
scope of this paper is Von Neumann Whitening \cite{vn_whitening} and basic bit
filtering. 

%add if needed

%Von Neumann Whitening is a basic post processing procedure that works with two
%bits at a time in three different ways: when two successive bits are equal,
%they are discarded; a sequence of 1,0 becomes a 1; and a sequence of 0,1
%becomes a zero.
 
Random numbers generated by our application will be compared to the stock
Android/Linux random number generator (which does not include an on-board
hardware RNG that may be present on newer mobile devices).  These sources will
be compared on three dimensions: quality of random numbers generated, energy
draw from using the random number generator, and computational resources used by
random number generation.  The NIST \cite{nist} benchmarks will be used to
evaluate the quality of the random numbers generated; synthetic benchmarks will
be written to evaluate the entropy rate and the computational complexity of the
random number generators.
