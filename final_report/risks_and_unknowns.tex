%% Risks and Unknowns

% Talk about:
%  - Is the accelerometer really entropic?
%  - Will this be worthwhile vs. the hw-rng?
%  - Is this data being incorporated somehow already?
%  - Unknown whether post processing is ``good''.

For any project, it is important to understand that there are risks and unknowns
that will need to be understood and mitigated during the course of the project.
By assessing what may become a problem during the process of the project, those
problems can be consciously avoided or dealt with early on.

One unknown that is central to this project is the entropy of the accelerometer.
Based on the findings in \cite{voris}, it appears that accelerometers are good
sources of entropy, it is possible that the accelerometer on the Dragonboard
violates some assumption or that technology has changed such that modern
accelerometers are no longer good sources of entropy. This can be tested using
several randomness tests, such as the DIEHARD and NIST entropy tests, which will
be essential to do before moving on with any sort of analysis.  These tests will
show if the sensor data appears to have good entropy or not.  In the case that
it does not, it would invalidate the remaining work for the project. 

A potential problem is the existence of the hardware random number generator
that is already on the board. Depending on how good that hardware is, it could
potentially dwarf any results that this project generates. Even so, the results
should be valuable, both as a look at the prospect of using accelerometers in
future designs for hardware random number generators as well as for providing
better entropy to existing platforms which do not have hardware random number
generators. 

During the testing phase of the project, members of the project team will
perform an experiment to show what sort of data is experienced over the course
of a couple of days. As it is not reasonable to carry the Dragonboard over the
course of a daily routine, some of the data will need to be collected on
personal cell phones. It could potentially be a problem if no available cell
phone has a similar accelerometer or gyroscope as this would introduce an
additional variable into the analysis. Research and experimentation will have to
be done to discover if, and show, the sensors in the selected cell phones will
be a reasonable stand in for the Dragonboard. If nothing matches closely enough,
a different method of collecting longer term real world data will have to be
devised.

If the data from these sensors is already being incorporated into the entropy
pool, it would make this work largely a repetition of that previous work. Upon
first inspection this does not appear to be the case, though further research is
needed to ensure that this is indeed novel work.
